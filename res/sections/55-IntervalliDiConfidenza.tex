\section{Intervalli di Confidenza}
Aggiunge informazione alla stima di un parametro, associando una misura di 
precisione. L'intervallo di precisione è da leggersi come  ``$(1-\alpha) \% $ 
degli intervalli numerici calcolabili conterrà il valore calcolato.''\\
Per campioni di numero ridotto ($\le$ 20) si può considerare:
\[ \frac{\hat{\beta_1} - \beta_1}{SE(\hat{\beta_1})} \sim t_{n-2} \]
Ovvero che si distribuisca come una t di Student con n-2 gradi di libertà.
Quindi l'intervallo di confidenza è:
\[ [ \hat{\beta_1} - t_{\alpha/2, n-2} SE(\hat{\beta_1}) ; \hat{\beta_1} + 
t_{\alpha/2, n-2} SE(\hat{\beta_1}) ] \]
Dove $t_{\alpha/2, n-2}$ è un quantile di livello $\alpha/2$ di una $t_{n-2}$
Per campioni di numero elevati ($\ge$ 20) si pu\`o considerare:
\[ \frac{\hat{\beta_1} - \beta_1}{SE(\hat{\beta_1})} \sim N(0,1) \]
Quindi l'intervallo di confidenza è:
\[ [ \hat{\beta_1} - z_{\alpha/2} SE(\hat{\beta_1}) ; \hat{\beta_1} +  
z_{\alpha/2} SE(\hat{\beta_1}) ] \]
Dove $z_{\alpha/2}$ è un quantile di livello $\alpha/2$ di una $N(0,1)$
\\ Per $\beta_0$ si procede in modo uguale.
