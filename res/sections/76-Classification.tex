\section{Classificazione}

In questo caso la response variabile Y è chiamata \textbf{categorica} ed è
qualitativa.

Predirre una variabile categorica ha lo scopo di assegnare una osservazione
a una categoria/classe.

Esempio: predirre se si verifichera un evento climatico disastroso date alcune
condizioni come temperatura, pressione, vento ecc\dots

Nel caso di una response variable binaria non è possibile usare il metodo di
regressione lineare poichè i valori che assume non sono reali ma discreti (0 o
1).

Una ``soluzione'' consiste nell'usare una funzione che restituisca valori reali
solamente nell'intervallo [0,1]. Il modello comunemente utilizzato per questo
scopo è il \textbf{modello di regressione logistica}.

\begin{equation}
P(Y=1|X) = \frac{e^{\beta_0+\beta_1 X}}{1 + e^{\beta_0+\beta_1 X}}
\end{equation}

Questa funzione assume valori in [0,1] indipendentemente da $\beta_0$ e
$\beta_1$.

Gli \textbf{odds} sono definiti dalla funzione:

\begin{equation}
\frac{P(Y=1|X)}{1-P(Y=1|X)} = e^{\beta_0+\beta_1 X}
\end{equation}

Valori vicini a 0 $\rightarrow$ probabilità molto bassa che Y=1.

Valori vicini a 1 $\rightarrow$ probabilità molto alta che Y=1.

I \textbf{logit} o \textbf{logodds} applicano il logaritmo agli odds:

\begin{equation}
\log \left( \frac{P(Y=1|X)}{1-P(Y=1|X)} \right) = \beta_0+\beta_1 X
\end{equation}

Questo termine \textbf{logit} è lineare in X.








